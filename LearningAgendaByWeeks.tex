\section {Week Learning Accomplishments}

\section  {Week 01}			

\begin{itemize}
\item Starting to learn what relational databases are. What is SQL? What is the Relational Database Model and how does it work?
\item Tables and Fields. Topics and Facts.
\item How does the relational database model compare with the Big Data Model?
\item What do Codd's Laws teach us? What did Codd's Laws do to advance the Cause of Civilization?
\item Primary Keys and Foreign Keys
\begin{itemize}
\item There are 6 Forms of Normalization.
\item Up to now, we have only talked about the First Normal Form: 1NF.
\item 1NF: Table A has a COLUMN with values that relate meaningfully to a COLUMN in Table B.
\item We create Predicate Joins by a few different methods. See the section: JOINING TABLES 
\item STUDY RESOURCE: \url{https://www.studytonight.com/dbms/database-normalization.php}
\end{itemize}

\begin{itemize}
\item The main reason for primary and foreign keys is to enforce data consistency.
\item A primary key enforces the consistency of uniqueness of values over one or more columns. If an ID column has a primary key then it is impossible to have two rows with the same ID value. Without that primary key, many rows could have the same ID value and you wouldn't be able to distinguish between them based on the ID value alone.
\item A foreign key enforces the consistency of data that points elsewhere. It ensures that the data which is pointed to actually exists. In a typical parent-child relationship, a foreign key ensures that every child always points at a parent and that the parent actually exists. Without the foreign key you could have "orphaned" children that point at a parent that doesn't exist.
\end{itemize}
\item Normalization and Minimization
\item 2 kinds of Informatics work: Analytics and Modeling
\item 4 kinds of SQL: Data Query Language, Data Manipulation Language, Data Control Language, Data Definition Language
\item Create Read Update Delete
\end{itemize}

\section {Week 02}
    
\begin{enumerate}
    \item Continuing the College Enrollment project
    \item Building up your Journal of SQL Statements in GitHub
     \item Using DDL to create Tables   
     \item Creating Scripts of SQL Statements   
     \item Using VIEWS
      \item Data Analytics : Creating Real Estate Market Reports
\end{enumerate}

\section {Week 03}
\subsection * {Farmer Joe SQL Exercise: Creating the Database and Writing the SQL}
\begin{enumerate}
    \item Data Definition Language: Creating the Tables
    \item Data Manipulation Language: Creating the Data
    \item Data Query Language
    \begin{itemize}
        \item Group By, Having, Aggregating Functions: Sum, Average, Max, Min 
        \item Using Views as Tables
        \item 4 kinds of joins: Left, Right, Inner, Outer
    \end{itemize} 
\end{enumerate}
    
\newpage
\section * {Week 04 Fill out Worksheet}
\textbf{}Week 04 Fill out Worksheet. \vspace{1cm} 
Name:   \vspace{1cm} 
Date:   \vspace{1cm} 
Class ID: \vspace{1cm}             \\
\begin{table}[]
\begin{tabular}{l|}
What does it mean to say that we do a PREDICATE JOIN on 2 SQL TABLES?  \vspace{1in}    \\

What are the differences between an INNER JOIN and a LEFT JOIN   \vspace{1in}       \\

What does an OUTER JOIN do?    \vspace{1in}      \\

How do we join tables using 1NF    \vspace{1in}     \\

Write a Query that uses a VIEW as a TABLE    \vspace{1in}     \\

Why do you LOVE 1NF ?  \vspace{1in}     \\

What are the most important elements of SQL and the Relational Database Model \\ as they relate to use as Data Modeling Specialists and Informatics Reporting Specialists? \\
For top marks, refer to Codd's Laws and 1NF    \vspace{3in}     \\

Discuss some of the principles of Information Systems Design that we have used in Class Exercises:   \vspace{3in}     \\



\end{tabular}
\end{table}

	
\iffalse

SQL Statements: Inner Joins

SQL Statements: Left Joins

SQL Statements: Right Joins

SQL Statements: Outer Joins

Creating Data Models: Joining Tables with Primary Keys and Predicate Joins

\fi 