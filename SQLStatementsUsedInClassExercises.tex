\section {This is a collection of SQL we developed in class exercises}

\begin{verbatim}
    INSERT INTO students ( studentid, studentlastname, studentfirstname, programgroupname ) VALUES ( 'A101', 'Smith', 'John', 'CPCT');
INSERT INTO students ( studentid, studentlastname, studentfirstname, programgroupname ) VALUES ( 'A102', 'Jones', 'Mike', 'CPCT');
INSERT INTO students ( studentid, studentlastname, studentfirstname, programgroupname ) VALUES ( 'A103', 'Smith', 'Susan', 'CPCT');
INSERT INTO students ( studentid, studentlastname, studentfirstname, programgroupname ) VALUES ( 'A104', 'Bichon', 'Peanut', 'CPCT');
INSERT INTO students ( studentid, studentlastname, studentfirstname, programgroupname ) VALUES ( 'A105', 'Smith', 'John', 'CPCT');
INSERT INTO students ( studentid, studentlastname, studentfirstname, programgroupname ) VALUES ( 'A106', 'Kirk', 'James', 'CPCT');
INSERT INTO students ( studentid, studentlastname, studentfirstname, programgroupname ) VALUES ( 'A107', 'Archer', 'Jonathan', 'CPCT');
INSERT INTO students ( studentid, studentlastname, studentfirstname, programgroupname ) VALUES ( 'A108', 'Janeway', 'Catherine', 'CPCT');
INSERT INTO students ( studentid, studentlastname, studentfirstname, programgroupname ) VALUES ( 'A109', 'Sisko', 'Benjamin', 'CPCT');
INSERT INTO students ( studentid, studentlastname, studentfirstname, programgroupname ) VALUES ( 'A110', 'Pike', 'Christopher', 'CPCT');
INSERT INTO students ( studentid, studentlastname, studentfirstname, programgroupname ) VALUES ( 'A111', 'Scott', 'Montgommery', 'CPCT');
INSERT INTO students ( studentid, studentlastname, studentfirstname, programgroupname ) VALUES ( 'A112', 'Riker', 'William', 'CPCT');


How to create the students table with DDL:
CREATE TABLE students(
studentid VARCHAR (20) NOT NULL,
studentlastname VARCHAR (20) NOT NULL,
studentfirstname VARCHAR (20) NOT NULL,
programgroupname VARCHAR (20) NOT NULL,
term VARCHAR (20) NOT NULL
PRIMARY KEY (studentid)
);

CREATE TABLE classes(
classid VARCHAR (20) NOT NULL,
roomid VARCHAR (20) NOT NULL,
datetimeid VARCHAR (20) NOT NULL,
coursename VARCHAR (20) NOT NULL
PRIMARY KEY (classid)
);

CREATE TABLE programgroup(
pgid VARCHAR (20) NOT NULL,
programgroupname VARCHAR(20) NOT NULL,
sectionnumber VARCHAR (20) NOT NULL,
classid VARCHAR (20) NOT NULL
PRIMARY KEY (pgid)
);

How to rename a COLUMN
USE CESVERSION2;
GO
EXEC sp_rename 'students.studentgroup' , 'programgroup';
GO


How to subtract the COUNTS of 2 Tables
DECLARE @ThisYearCount BIGINT,
        @LastYearCount BIGINT,
        @Difference  BIGINT

SELECT @ThisYearCount = COUNT(*) FROM ThisYear

SELECT @LastYearCount = COUNT(*) FROM LastYear

SET @Difference = @ThisYearCount - @LastYearCount

Print @Difference


How to delete a Column from a Table:

ALTER TABLE Customers
DROP COLUMN ContactName;


create view CombinedData as 

select this.Description, this.Type, this.bedroom , c.communityname, c.communitydata

from ThisYear as this, communityinfo as c 

where [sold price] < 600000 

and this.Community = c.communityname;

select * from CombinedData as cv where cv.bedroom  >3


WEEK 3 Code:

Farmer Joe Exercise:

CREATE TABLE veggies (
    veggie_id varchar(20),
    cost money,
    veggie_name varchar(20),
    quantity int
    PRIMARY KEY (veggie_id)
);

CREATE TABLE lot (
    lot_id varchar(20),
    veggie_id varchar(20),
    lot_name varchar(20),
    PRIMARY KEY (lot_id)
);




CREATE TABLE veggies(
veggie_id varchar(20),
veggie_price money,
veggie_name varchar(20),
veggie_type varchar(20)
PRIMARY KEY (veggie_id)
);

How to print out all fields in ALL Tables in a Database:

 SELECT TABLE_SCHEMA ,
       TABLE_NAME ,
       COLUMN_NAME ,
       ORDINAL_POSITION ,
       COLUMN_DEFAULT ,
       DATA_TYPE ,
       CHARACTER_MAXIMUM_LENGTH ,
       NUMERIC_PRECISION ,
       NUMERIC_PRECISION_RADIX ,
       NUMERIC_SCALE ,
       DATETIME_PRECISION
FROM   INFORMATION_SCHEMA.COLUMNS;
\end{verbatim}

\begin{verbatim}
    

This example uses an expression and gives it a column alias.

SELECT
first_name,
last_name,
first_name || ' ' || last_name AS full_name
FROM employee_test;

This example uses a column alias on a function.

SELECT
last_name,
hdt AS hire_date,
ROUND(MONTHS_BETWEEN(SYSDATE, hdt),0) AS tenure_months
FROM employee_test;

\end{verbatim}