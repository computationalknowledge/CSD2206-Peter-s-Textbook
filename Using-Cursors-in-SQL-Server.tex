\newline
\section {Using Cursor in SQL Server Stored Procedures}
% https://www.aspsnippets.com/Articles/Using-Cursor-in-SQL-Server-Stored-Procedure-with-example.aspx

What are Cursors in SQL Server?
A CURSOR IS A PACKAGING MECHANISM THAT ALLOWS US TO MOVE DATA FROM ON QUERY TO ANOTHER

Cursor is a Database object which allows us to process each row and manipulate its data. A Cursor is always associated with a Select Query and it will process each row returned by the Select Query one by one.
Using Cursor we can verify each row data, modify it or perform calculations which are not possible when we get all records at once.
A simple example would be a case where you have records of Employees and you need to calculate Salary of each employee after deducting Taxes and Leaves.
 
 
What is the syntax for writing Cursors in SQL Server?
Below is the syntax for writing a Cursor. The very first thing is to declare some variables based on the columns you are fetching in your Select Query.
Note: In this article I am considering only READ_ONLY Cursors.
Then you need to declare the Cursor by giving it a name and setting its type as READ_ONLY and along with the FOR keyword you need to write the Select Query which will return the records you need to process.
Once the Cursor is setup, we need to open it using the OPEN command and then the first record is fetched and saved into the variable.
Whenever a record is fetched the @@FETCH_STATUS has value 0 and as soon as all the records returned by the Select Query are fetched, its value changes to -1.
A Cursor is associated with a WHILE LOOP which executes until the @@FETCH_STATUS has value 0.
Inside the WHILE LOOP, the processing is done for the current record and then again the next record is fetched and this process continues until @@FETCH_STATUS is 0.
Finally the Cursor is closed and deallocated using CLOSE and DEALLOCATE commands respectively.
Note: It is very important to DEALLOCATE a Cursor as otherwise it will stay in database and when you declare a Cursor with same name again, SQL Server will throw an error: A cursor with the name 'Cursor1' already exists.
 
Using Cursor in SQL Server Stored Procedure with example
 
 
How to write and use Cursors in SQL Server Stored Procedure?
Following is an example of simple Cursor in SQL Server Stored Procedure which prints all the records of Customers table
SET ANSI_NULLS ON
GO
SET QUOTED_IDENTIFIER ON
GO
CREATE PROCEDURE PrintCustomers_Cursor
AS
BEGIN
      SET NOCOUNT ON;
 
      --DECLARE THE VARIABLES FOR HOLDING DATA.
      DECLARE @CustomerId INT
             ,@Name VARCHAR(100)
             ,@Country VARCHAR(100)
 
      --DECLARE AND SET COUNTER.
      DECLARE @Counter INT
      SET @Counter = 1
 
      --DECLARE THE CURSOR FOR A QUERY.
      DECLARE PrintCustomers CURSOR READ_ONLY
      FOR
      SELECT CustomerId, Name, Country
      FROM Customers
 
      --OPEN CURSOR.
      OPEN PrintCustomers
 
      --FETCH THE RECORD INTO THE VARIABLES.
      FETCH NEXT FROM PrintCustomers INTO
      @CustomerId, @Name, @Country
 
      --LOOP UNTIL RECORDS ARE AVAILABLE.
      WHILE @@FETCH_STATUS = 0
      BEGIN
             IF @Counter = 1
             BEGIN
                        PRINT 'CustomerID' + CHAR(9) + 'Name' + CHAR(9) + CHAR(9) + CHAR(9) + 'Country'
                        PRINT '------------------------------------'
             END
 
             --PRINT CURRENT RECORD.
             PRINT CAST(@CustomerId AS VARCHAR(10)) + CHAR(9) + CHAR(9) + CHAR(9) + @Name + CHAR(9) + @Country
    
             --INCREMENT COUNTER.
             SET @Counter = @Counter + 1
 
             --FETCH THE NEXT RECORD INTO THE VARIABLES.
             FETCH NEXT FROM PrintCustomers INTO
             @CustomerId, @Name, @Country
      END
 
      --CLOSE THE CURSOR.
      CLOSE PrintCustomers
      DEALLOCATE PrintCustomers
END
GO
 
The following screenshot displays the records printed by the above Cursor.
Using Cursor in SQL Server Stored Procedure with example
You can also checkout an advanced SQL Server Cursor that sends email in my article Automated Email Notifications using SQL Server Job Scheduler.
 
 
Disadvantages of Cursor
The major disadvantage of a Cursor is its performance issue. A Cursor can be really slow when performing operations on large number of records and your SQL Query may take minutes to execute and produce results.

