\section{Joining Tables}

Predicate Joins are a sub set of CONDITIONS in SQL Clauses.

\section{Ways of joining tables:}
\url{https://www.sql-join.com/sql-join-types/}

\subsection{Is there any difference (performance, best-practice, etc...) between putting a condition in the JOIN clause vs. the WHERE clause?}

\begin{verbatim}
    -- Condition in JOIN
SELECT *
FROM dbo.Customers AS CUS
INNER JOIN dbo.Orders AS ORD 
ON CUS.CustomerID = ORD.CustomerID
AND CUS.FirstName = 'John'

-- Condition in WHERE
SELECT *
FROM dbo.Customers AS CUS
INNER JOIN dbo.Orders AS ORD 
ON CUS.CustomerID = ORD.CustomerID
WHERE CUS.FirstName = 'John'

Which do you prefer (and perhaps why)?

\end{verbatim}