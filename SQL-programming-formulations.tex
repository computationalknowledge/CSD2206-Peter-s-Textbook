\newline
\section {SQL Programming Formulations}

\section {Using the PRINT command}
\begin{lstlisting}[frame=single]
PRINT 'Hello World';
DECLARE @myvariable VARCHAR(40) = 'I like essentialSQL.'
PRINT @myvariable
DECLARE @row INT = 63;
PRINT 'The value of row is ' + CAST(@row as VARCHAR);
\end{lstlisting}

\begin{lstlisting}[frame=single]

How to Use IF...THEN Logic in SQL Server

SQL Server has a unique capability of allowing you to execute real-time programmatic logic on the values within your query. Based on those logical evaluations, you can generate various values as part of the returned data set.

Using the CASE Statement
This is most easily accomplished in all versions of SQL Server using the CASE statement, which acts as a logical IF...THEN...ELSE expression and returns various values depending on the result.

In this example below, we want to return an additional locale column that specifies whether our book takes place in Middle-earth or regular old Earth.

SELECT
  CASE
    WHEN
      books.title = 'The Hobbit'
        THEN
          'Middle-earth'
    WHEN
      books.primary_author = 'Tolkien'
        THEN
          'Middle-earth'
    ELSE
      'Earth'
  END AS locale,
  books.*
FROM
  books
Before we examine the special CASE aspect of this statement, let’s temporarily remove the CASE to notice that this is an extremely simple SELECT statement on the surface:

SELECT
  books.*
FROM
  books
Therefore, let’s examine how the CASE section is structured and what logical behavior we’re performing.

CASE
  WHEN
    books.title = 'The Hobbit'
      THEN
        'Middle-earth'
  WHEN
    books.primary_author = 'Tolkien'
      THEN
        'Middle-earth'
  ELSE
    'Earth'
END AS locale
To begin, we of initialize the CASE statement then specify under which conditions (WHEN) our CASE statement should evaluate a result. In this example, we’re examining the books.title and books.primary_author; if either fit our Tolkien-esque theme, THEN we return the value ‘Middle-earth.’ If neither fields match our search, we instead return the value of ‘Earth.’

To rearrange the logic as a psuedo-code IF...THEN...ELSE statement, we’re simply asking SQL to evaluate:

IF
  title == 'The Hobbit' OR
  primary_author == 'Tolkien'
THEN
  RETURN 'Middle-earth'
ELSE
  RETURN 'Earth'
END
Finally, it is critical to remember that a CASE statement must always be appended at the end with a matching END statement. In the above example, we’re also renaming the resulting value that is returned to locale, though that is certainly optional.

Using the IIF Function
If you are using a more modern version of SQL, it is useful to know that SQL Server 2012 introduced the very handy IIF function. IIF is a shorthand method for performing an IF...ELSE/CASE statement and returning one of two values, depending on the evaluation of the result.

Restructuring our above example to use IIF is quite simple.

SELECT
  IIF(
    books.title = 'The Hobbit' OR books.primary_author = 'Tolkien',
    'Middle-earth',
    'Earth')
  AS locale,
  books.*
FROM
  books
With an IIF function, we largely replace a lot of the syntactical sugar from the CASE statement with a few simple comma-seperators to differentiate our arguments.

Both CASE and IIF get the same job done, but if given the choice, IIF will generally be much simpler to use.

\end{lstlisting}

\newpage

\begin{lstlisting}[frame=single]

SQL Server: FOR LOOP
Learn how to simulate the FOR LOOP in SQL Server (Transact-SQL) with syntax and examples.

TIP: Since the FOR LOOP does not exist in SQL Server, this page describes how to simulate a FOR LOOP using a WHILE LOOP.
Description
In SQL Server, there is no FOR LOOP. However, you simulate the FOR LOOP using the WHILE LOOP.

Syntax
The syntax to simulate the FOR Loop in SQL Server (Transact-SQL) is:

DECLARE @cnt INT = 0;

WHILE @cnt < cnt_total
BEGIN
   {...statements...}
   SET @cnt = @cnt + 1;
END;
Parameters or Arguments
cnt_total
The number of times that you want the simulated FOR LOOP (ie: WHILE LOOP) to execute.
statements
The statements of code to execute each pass through the loop.
Note
You can simulate the FOR LOOP in SQL Server (Transact-SQL) using the WHILE LOOP.
Example
Let's look at an example that shows how to simulate the FOR LOOP in SQL Server (Transact-SQL) using the WHILE LOOP.

For example:

DECLARE @cnt INT = 0;

WHILE @cnt < 10
BEGIN
   PRINT 'Inside simulated FOR LOOP on TechOnTheNet.com';
   SET @cnt = @cnt + 1;
END;

PRINT 'Done simulated FOR LOOP on TechOnTheNet.com';
GO
In this WHILE LOOP example, the loop would terminate once @cnt reaches 10.


\end{lstlisting}    