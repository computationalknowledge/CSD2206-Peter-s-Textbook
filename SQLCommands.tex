\section * {SQL Commands}

\begin{verbatim}
  
SQL Statement	Syntax
AND / OR	SELECT column_name(s)
FROM table_name
WHERE condition
AND|OR condition
ALTER TABLE	ALTER TABLE table_name
ADD column_name datatype
or

ALTER TABLE table_name
DROP COLUMN column_name

AS (alias)	SELECT column_name AS column_alias
FROM table_name
or

SELECT column_name
FROM table_name  AS table_alias

BETWEEN	SELECT column_name(s)
FROM table_name
WHERE column_name
BETWEEN value1 AND value2
CREATE DATABASE	CREATE DATABASE database_name
CREATE TABLE	CREATE TABLE table_name
(
column_name1 data_type,
column_name2 data_type,
column_name3 data_type,
...
)
CREATE INDEX	CREATE INDEX index_name
ON table_name (column_name)
or

CREATE UNIQUE INDEX index_name
ON table_name (column_name)

CREATE VIEW	CREATE VIEW view_name AS
SELECT column_name(s)
FROM table_name
WHERE condition
DELETE	DELETE FROM table_name
WHERE some_column=some_value
or

DELETE FROM table_name
(Note: Deletes the entire table!!)

DELETE * FROM table_name
(Note: Deletes the entire table!!)

DROP DATABASE	DROP DATABASE database_name
DROP INDEX	DROP INDEX table_name.index_name (SQL Server)
DROP INDEX index_name ON table_name (MS Access)
DROP INDEX index_name (DB2/Oracle)
ALTER TABLE table_name
DROP INDEX index_name (MySQL)
DROP TABLE	DROP TABLE table_name
EXISTS	IF EXISTS (SELECT * FROM table_name WHERE id = ?)
BEGIN
--do what needs to be done if exists
END
ELSE
BEGIN
--do what needs to be done if not
END
GROUP BY	SELECT column_name, aggregate_function(column_name)
FROM table_name
WHERE column_name operator value
GROUP BY column_name
HAVING	SELECT column_name, aggregate_function(column_name)
FROM table_name
WHERE column_name operator value
GROUP BY column_name
HAVING aggregate_function(column_name) operator value
IN	SELECT column_name(s)
FROM table_name
WHERE column_name
IN (value1,value2,..)
INSERT INTO	INSERT INTO table_name
VALUES (value1, value2, value3,....)
or

INSERT INTO table_name
(column1, column2, column3,...)
VALUES (value1, value2, value3,....)

INNER JOIN	SELECT column_name(s)
FROM table_name1
INNER JOIN table_name2
ON table_name1.column_name=table_name2.column_name
LEFT JOIN	SELECT column_name(s)
FROM table_name1
LEFT JOIN table_name2
ON table_name1.column_name=table_name2.column_name
RIGHT JOIN	SELECT column_name(s)
FROM table_name1
RIGHT JOIN table_name2
ON table_name1.column_name=table_name2.column_name
FULL JOIN	SELECT column_name(s)
FROM table_name1
FULL JOIN table_name2
ON table_name1.column_name=table_name2.column_name
LIKE	SELECT column_name(s)
FROM table_name
WHERE column_name LIKE pattern
ORDER BY	SELECT column_name(s)
FROM table_name
ORDER BY column_name [ASC|DESC]
SELECT	SELECT column_name(s)
FROM table_name
SELECT *	SELECT *
FROM table_name
SELECT DISTINCT	SELECT DISTINCT column_name(s)
FROM table_name
SELECT INTO	SELECT *
INTO new_table_name [IN externaldatabase]
FROM old_table_name
or

SELECT column_name(s)
INTO new_table_name [IN externaldatabase]
FROM old_table_name

SELECT TOP	SELECT TOP number|percent column_name(s)
FROM table_name
TRUNCATE TABLE	TRUNCATE TABLE table_name
UNION	SELECT column_name(s) FROM table_name1
UNION
SELECT column_name(s) FROM table_name2
UNION ALL	SELECT column_name(s) FROM table_name1
UNION ALL
SELECT column_name(s) FROM table_name2
UPDATE	UPDATE table_name
SET column1=value, column2=value,...
WHERE some_column=some_value
WHERE	SELECT column_name(s)
FROM table_name
WHERE column_name operator value

 \end{verbatim}